\subsection{Interpolation}\label{sec:interpolation}

These functions are for interpolating step functions
given by pairs of $x$- and $y$-values at specified
values.


%%%%%
\subsection*{abs-differences-for-curves-at-points}\label{fun:abs-differences-for-curves-at-points}

\vspace{0.3cm}
\begin{tabular}{r|p{8cm}}
Started, last checked & 1/10/2010, 1/10/2010 \\
Location & \nameref{sec:interpolation} \\
Calls & \nameref{fun:linearly-interpolate} \\
Called by & \\
Comments/see also & 
\end{tabular}

\vspace{0.5cm}
\noindent Example:
\begin{verbatim}
(setq knot-value-pairs1 '((0 0) (1.5 1) (2 3) (4 2)))
(setq knot-value-pairs2 '((0 0) (1 1) (2 3) (4 3)))
(abs-differences-for-curves-at-points
 knot-value-pairs1 knot-value-pairs2)
--> (0 1.0 0 1)
\end{verbatim}

\noindent Two lists of knot-value pairs are provided
as arguments. The $x$-values of the first argument are
interpolated using the second argument. The absolute
difference between these interpolated values and the
actual $y$-values of the first argument is
returned.


%%%%%
\subsection*{index-1st-sublist-item$<$}\label{fun:index-1st-sublist-item<}

\vspace{0.3cm}
\begin{tabular}{r|p{8cm}}
Started, last checked & 1/10/2010, 1/10/2010 \\
Location & \nameref{sec:interpolation} \\
Calls & \\
Called by & \\
Comments/see also & \nameref{fun:index-nth-sublist-item<}, 
\newline \nameref{fun:index-1st-sublist-item<=}
\end{tabular}

\vspace{0.5cm}
\noindent Example:
\begin{verbatim}
(index-1st-sublist-item<
 0 '(14 14 14 11 0 0 -1 -2 -2))
--> 6
\end{verbatim}

\noindent This function takes two arguments: a real
number $x$ and a list $L$ of real numbers. It returns
the index of the first element of $L$ which is less
than $x$.


%%%%%
\subsection*{index-1st-sublist-item$>$}\label{fun:index-1st-sublist-item>}

\vspace{0.3cm}
\begin{tabular}{r|p{8cm}}
Started, last checked & 1/10/2010, 1/10/2010 \\
Location & \nameref{sec:interpolation} \\
Calls & \\
Called by & \nameref{fun:linearly-interpolate} \\
Comments/see also & \nameref{fun:index-nth-sublist-item>}, 
\newline \nameref{fun:index-1st-sublist-item>=}
\end{tabular}

\vspace{0.5cm}
\noindent Example:
\begin{verbatim}
(index-1st-sublist-item>
 0 '(-2 -2 -1 0 0 11 14 14 14))
--> 5
\end{verbatim}

\noindent This function takes two arguments: a real
number $x$ and a list $L$ of real numbers. It returns
the index of the first element of $L$ which is greater
than $x$.


%%%%%
\subsection*{linearly-interpolate}\label{fun:linearly-interpolate}

\vspace{0.3cm}
\begin{tabular}{r|p{8cm}}
Started, last checked & 1/10/2010, 1/10/2010 \\
Location & \nameref{sec:interpolation} \\
Calls & \nameref{fun:index-1st-sublist-item>}, \nameref{fun:my-last}, \newline \nameref{fun:nth-list-of-lists} \\
Called by & \nameref{fun:abs-differences-for-curves-at-points}, \nameref{fun:linearly-interpolate-x-values} \\
Comments/see also & 
\end{tabular}

\vspace{0.5cm}
\noindent Example:
\begin{verbatim}
(setq knot-value-pairs '((0 0) (1 1) (2 3) (4 3)))
(linearly-interpolate 1.5 knot-value-pairs)
--> 2.0
\end{verbatim}

\noindent The second argument is a list of knot-value
pairs. The $x$-value of the first argument is
interpolated to give a $y$-value using the knot-value
pairs. If the first argument exceeds the endpoints,
the appropriate endpoint value is returned.


%%%%%
\subsection*{linearly-interpolate-x-values}\label{fun:linearly-interpolate-x-values}

\vspace{0.3cm}
\begin{tabular}{r|p{8cm}}
Started, last checked & 1/10/2010, 1/10/2010 \\
Location & \nameref{sec:interpolation} \\
Calls & \nameref{fun:linearly-interpolate} \\
Called by & \\
Comments/see also & 
\end{tabular}

\vspace{0.5cm}
\noindent Example:
\begin{verbatim}
(setq knot-value-pairs '((0 0) (1 1) (2 3) (4 3)))
(linearly-interpolate-x-values
 '(1.5 2 1.75) knot-value-pairs)
--> (2.0 3 2.5)
\end{verbatim}

\noindent The second argument is a list of knot-value
pairs. The first argument is a list of $x$-values that
are interpolated to give $y$-values using the knot-
value pairs. If any members of the first argument
exceeds the endpoints, the appropriate endpoint value
is returned.











