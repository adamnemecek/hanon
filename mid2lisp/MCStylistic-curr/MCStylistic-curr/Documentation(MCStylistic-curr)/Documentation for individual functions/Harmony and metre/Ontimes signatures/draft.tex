\subsection{Ontimes signatures}\label{sec:ontimes-signatures}

These lisp functions help with converting between
ontimes and bar/beat numbers, when time signatures
change according to the variable time-sigs-with-
ontimes, which is a list of time signatures in a
piece. In this list of lists: the first item of a
list is a bar number; the second item is the upper
number of the time signature in that bar (specifying
number of beats per bar); the third item is the
lower number of the time signature (specifying the
division used to count time, with 4 for crotchet,
etc.); the fourth item is the corresponding
ontime.


%%%%%
\subsection*{append-ontimes-to-time-signatures}\label{fun:append-ontimes-to-time-signatures}

\vspace{0.3cm}
\begin{tabular}{r|p{8cm}}
Started, last checked & 15/3/2010, 15/3/2010 \\
Location & \nameref{sec:ontimes-signatures} \\
Calls & \nameref{fun:my-last} \\
Called by & \\
Comments/see also & \nameref{fun:kern-file2ontimes-signatures}
\end{tabular}

\vspace{0.5cm}
\noindent Example:
\begin{verbatim}
(append-ontimes-to-time-signatures
 '((1 2 4) (2 3 8) (4 3 4) (7 5 8)))
--> ((1 2 4 0) (2 3 8 2) (4 3 4 5) (7 5 8 14))
\end{verbatim}

\noindent This function appends ontimes to rows of
the time-signature table.


%%%%%
\subsection*{bar\&beat-number-of-ontime}\label{fun:bar-n-beat-number-of-ontime}

\vspace{0.3cm}
\begin{tabular}{r|p{8cm}}
Started, last checked & 15/3/2010, 10/6/2015 \\
Location & \nameref{sec:ontimes-signatures} \\
Calls & \nameref{fun:row-of-max-ontime<=ontime-arg} \\
Called by & \nameref{fun:bar-beat-ontimes} \\
Comments/see also & 
\end{tabular}

\vspace{0.5cm}
\noindent Example:
\begin{verbatim}
(bar&beat-number-of-ontime
 10 '((1 2 4 0) (2 3 8 2) (4 3 4 5) (7 5 8 14)))
--> (5 3 10)
\end{verbatim}

\noindent Given an ontime and a time-signature table
(with ontimes appended), this function returns the
bar number and beat number of that ontime. Beat
numbers are expressed in crotchets counting from one
(see the second and third examples above).


%%%%%
\subsection*{bar-beat-ontimes}\label{fun:bar-beat-ontimes}

\vspace{0.3cm}
\begin{tabular}{r|p{8cm}}
Started, last checked & 15/3/2010, 15/3/2010 \\
Location & \nameref{sec:ontimes-signatures} \\
Calls & \nameref{fun:bar-n-beat-number-of-ontime} \\
Called by & \nameref{fun:bar-beat-ontimes} \\
Comments/see also & 
\end{tabular}

\vspace{0.5cm}
\noindent Example:
\begin{verbatim}
(bar-beat-ontimes
 0 1 '((1 2 4 0) (2 3 8 2) (4 3 4 5) (7 5 8 14)))
--> ((1 1 0) (1 2 1) (2 1 2) (2 2 3) (3 3/2 4)
     (4 1 5) (4 2 6) (4 3 7) (5 1 8) (5 2 9)
     (5 3 10) (6 1 11) (6 2 12) (6 3 13) (7 1 14)
     (7 2 15))
\end{verbatim}

\noindent Given an ontime start, a subdivision of
the crotchet beat and a time-signature table (with
ontimes appended, this function returns the bar and
beat numbers that will be displayed at the foot of
the MIDI table, and the corresponding ontimes.


%%%%%
\subsection*{increment-by-x-n-times}\label{fun:increment-by-x-n-times}

\vspace{0.3cm}
\begin{tabular}{r|p{8cm}}
Started, last checked & 15/3/2010, 15/3/2010 \\
Location & \nameref{sec:ontimes-signatures} \\
Calls & \nameref{fun:my-last} \\
Called by & \nameref{fun:bar-beat-ontimes} \\
Comments/see also & 
\end{tabular}

\vspace{0.5cm}
\noindent Example:
\begin{verbatim}
(increment-by-x-n-times 1/2 3 7.5)
--> (7.5 8.0 8.5 9.0)
\end{verbatim}

\noindent Adds the first argument to the third
(default zero), and continues to do so until the
second argument is exceeded.


%%%%%
\subsection*{ontime-of-bar\&beat-number}\label{fun:ontime-of-bar-n-beat-number}

\vspace{0.3cm}
\begin{tabular}{r|p{8cm}}
Started, last checked & 15/3/2010, 15/3/2010 \\
Location & \nameref{sec:ontimes-signatures} \\
Calls & \nameref{fun:my-last} \\
Called by & \nameref{fun:bar-beat-ontimes} \\
Comments/see also & 
\end{tabular}

\vspace{0.5cm}
\noindent Example:
\begin{verbatim}
(ontime-of-bar&beat-number
 5 2 '((1 2 4 0) (2 3 8 2) (4 3 4 5) (8 5 8 17)))
--> (5 2 9)
\end{verbatim}

\noindent Given a bar and beat number, and a
time-signature table (with ontimes appended, this
function returns the ontime of that bar and beat
number.


%%%%%
\subsection*{row-of-max-bar$<$=ontime-bar}\label{fun:row-of-max-bar<=ontime-bar}

\vspace{0.3cm}
\begin{tabular}{r|p{8cm}}
Started, last checked & 15/3/2010, 15/3/2010 \\
Location & \nameref{sec:ontimes-signatures} \\
Calls & \\
Called by & \nameref{fun:ontime-of-bar-n-beat-number} \\
Comments/see also & 
\end{tabular}

\vspace{0.5cm}
\noindent Example:
\begin{verbatim}
(row-of-max-bar<=ontime-bar
 4 '((1 2 4 0) (2 3 8 2) (4 3 4 5) (8 5 8 17)))
--> (4 3 4 5)
\end{verbatim}

\noindent Returns the row (in a list of time
signatures) of the maximal bar number less than the
bar number argument.


%%%%%
\subsection*{row-of-max-ontime$<$=ontime-arg}\label{fun:row-of-max-ontime<=ontime-arg}

\vspace{0.3cm}
\begin{tabular}{r|p{8cm}}
Started, last checked & 15/3/2010, 2/1/2015 \\
Location & \nameref{sec:ontimes-signatures} \\
Calls & \\
Called by & \nameref{fun:bar-n-beat-number-of-ontime} \\
Comments/see also & 
\end{tabular}

\vspace{0.5cm}
\noindent Example:
\begin{verbatim}
(row-of-max-ontime<=ontime-arg
 7 '((1 2 4 0) (2 3 8 2) (4 3 4 5) (5 5 8 8)))
--> (4 3 4 5)
\end{verbatim}

\noindent Returns the row (in a list of time
signatures) of the maximal ontime less than the
ontime argument.

2/1/2015. Added handling of negative ontimes
(e.g., representing an anacrusis).


















